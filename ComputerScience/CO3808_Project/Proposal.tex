% Import class techplan
\documentclass[a4paper,11pt]{proposal}

% For full width tables
\usepackage{tabularx}
\usepackage{parskip}
\usepackage[xetex]{hyperref}
\usepackage{soul, color}

\pagenumbering{gobble}

\author{Your Full Name Goes Here}
\course{Full and Proper Name of Your Course Goes Here}

\title{The Project Title}

\begin{document}

\maketitle

\hl{Do not add any front sheets or title pages. Read these notes highlighted in yellow and remove before submitting. You should also delete this paragraph of text. Minimum page limit of 2 and maximum of 3 pages required. Around 1000 words.}

\section*{Project Context}

\hl{A summary of what you intend to achieve A brief description of the background and issues relevant to your project, which may include the benefits of undertaking the project and an outline of previous work in this area (e.g. other projects, information from clients/employers, results of an initial literature survey). Can be worthwhile thinking about your project as the solution to a problem: in which case state the problem and how your project solves it.}

\section*{Specific Objectives}

\hl{3-5 key objectives or specific outcomes.  This should include an indication of the capabilities of the product.}

\section*{Potential Ethical or Legal Issues}

\hl{List potential areas of concern or where special care may be needed. (e.g., working with children, personal information, or an external client. Please do not write ‘none’ in this section, your proposal will be bounced back to you if you do. Every project has some issues that need to be considered, even if they turn out not to be issues.}

\section*{Resources}

\hl{List resources (and alternatives if there is any uncertainty about specific resources) to show that sufficient resources (e.g. tools and any client time) are available.}

\section*{Potential Commercial Considerations - Estimated costs and benefits}

\hl{Briefly list the factors that would affect the economic viability of the proposed project, if it were done commercially. Identify sources of costs and benefits. As well as the costs and benefits, the timing of the project and features that make it distinctive may also be key factor in its commercial success (Something developed for the next Olympics will be worthless if it is late. Software with unique features is likely to be more successful). If you are developing software that would be sold, consider how you would make money from this. If you are developing software to support a company, consider the benefits it provides. You will analyse this in more detail in the technical plan. You are expected to put costings in here (i.e. numbers in £).}

\section*{Proposed Approach}
\hl{This is an initial estimation by you of the tasks you think will be involved in doing your project and how long you think each will take. Rough estimates are fine. You can include an outline of tools and techniques to be used.

Briefly list the stages or steps of the project development.}

\section*{References}

\hl{Six references to appropriate articles and web pages that are relevant to your project. At least 3 of the references must be to journal papers. References must be referred to in the body of your document. If you just list them here but do not ‘refer’ to them then they are not references, this becomes a bibliography instead, and that is not what is being asked for here. Your references must be given using the APA referencing style. If you do not know what this is then look it up. There are examples to help you on Blackboard. List alphabetically by author surname. These are very specific instructions, please follow them.}

\end{document}
