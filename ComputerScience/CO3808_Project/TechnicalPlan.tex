% Import class techplan
\documentclass[a4paper,11pt]{techplan}

% For full width tables
\usepackage{tabularx}

\usepackage[dvips]{hyperref}

\author{Your Full Name Goes Here}
\course{Full and Proper Name of Your Course Goes Here}
\supervisory{State here the supervisor you have been allocated to (if you don’t know then leave blank)}

\title{The Project Title}


\begin{document}

\maketitle

\section*{Summary}

Project Summary. Challenges - what makes this project difficult? Solutions - a summary of the key points of this plan. Specific objectives – what are you trying to achieve? A quick summary of the approach/methodology that you are taking.

\section*{Deliverables}

What will you actually be submitting? What will be your artefact(s).

\section*{Constraints}

Any external constraints on the project (e.g. the fixed project deadlines).

\section*{Key Problems}

The main difficulties that have to be overcome for a successful project - For a database system these might be related to: identifying the user needs, developing a data model, user interface design and implementation.  This section should clarify the main problems to allow the reader to check the solutions.

\section*{System and Work Outline}
\begin{flushleft}
What will be done, how it will be done and why.
\linebreak
\linebreak
\medskip
This may include:
\linebreak
Methods for analysis, design and implementation of system or experiment simulation/investigation.
\linebreak
Development/simulation/experimental Environment: languages and tools.
\linebreak
Development projects: identification of key requirements. High level design breaking the system into components and indicating techniques that will be used in the components. Where the techniques are unknown, this should be identified.  Key algorithms may be identified.
\linebreak
Research projects: identification of key aspects to be investigated and factors that affect these aspects. High level design of your experiment/simulation/investigation to show how you will ensure your conclusions are valid and generalisable and how you will limit interference from other factors. Key approaches to analysing results should be identified (e.g. if numerical data is being created, identify potentially relevant statistical techniques, which may be investigated further during the project.) Where the techniques are unknown, this should be identified.
If the project is to be formally evaluated, appropriate methods should be identified.
\linebreak
Known issues which remain to be resolved \& how they will be resolved.
\linebreak
Personal development: skills and knowledge that will need to be acquired.
\linebreak
Report issues - significant topics for the report \& how you will obtain relevant information
\linebreak
Significant Implications (e.g. expected limitations of the project)
\end{flushleft}

\section*{Project Activities}
List, prioritise and time plan the activities for your project. Depending on the approach you’re taking this may take the form or a project backlog and an outline the approach taken to timeboxes/sprints, or it may be a Gantt Chart.

\section*{Risk Analysis}
Significant potential hazards and how they will be managed. e.g. availability of necessary resources.  Do not simply add risks for the sake of it.  Consider using a tabular presentation:

\begin{center}
    \begin{tabularx}{\textwidth}{|X|X|X|X|}
    \hline
    \textbf{Risk} & \textbf{Severity} & \textbf{Likelihood} & \textbf{Action} \\[1ex]
    \hline
    & & & \\
    \hline
    & & & \\
    \hline
    \end{tabularx}
\end{center}

\section*{Options}

The alternatives considered in forming this plan and their advantages and disadvantages.  This may include:
Lifecycles, Development Methods, Development Tools, Target Environment.
Do not simply add options for the sake of it.

\section*{Potential Ethical or Legal Issues}

List potential areas of concern or where special care may be needed. (e.g., working with children, personal information, or an external client. Please do not write ‘none’ in this section, your proposal will be bounced back to you if you do. Every project has some issues that need to be considered, even if they turn out not to be issues. Consider any applicable health and safety matters. Think about diversity, inclusion, cultural, societal and environmental matters concerns.

\section*{Commercial Analysis}

Estimated costs and benefits
Assess the economic viability of the proposed project, if it were done commercially. You will need to identify sources of costs and benefits. Do not assume that your time is free (You do want to eat after you graduate, don’t you?) or that you could use University software to develop commercial products. Educational licences do not usually permit this. Where possible, identify numerical costs, so your supervisor can discuss them with you. The time when costs arise or when money is paid can be important – if you have to borrow money, you have to pay interest – increase costs by 10\% to account for this.

\begin{center}
    \begin{tabularx}{\textwidth}{|X|X|X|X|X|}
    
    \hline
    \textbf{Factor name} & \textbf{Description} & \textbf{Is this a cost or a benefit} & \textbf{Estimated Amount} & \textbf{Estimate of when paid} \\
    \hline
    & & & & \\
    \hline
    & & & & \\
    \hline
    \end{tabularx}
\end{center}

\section*{Employability Contribution}

How do you see this project aiding towards building your employability portfolio? This section should contain a description of the project and the skills that you have gained. Imagine this would be the section that you include in your CV portfolio and write it as such.

\section*{References}

Please include at least 8 references in APA format. Check them carefully so they are all following the same format. Use the guide available on Blackboard to get this correct as it may affect your grade. All references should be referred to in your technical plan (this does not necessarily mean ‘quoted’). You can reuse some from your proposal but at least a minimum of 3 new ones are expected. 

\pagebreak

\section*{Notes on Technical Planning (delete this section before submitting)}

\subsection*{Introduction}
The Technical Plan is an overview of how the project will be carried out discussing, defining, and justifying:

\begin{itemize}
    \item the approach to the project (lifecycle details, specific activities and dependencies, decision points)
    \item the tools and techniques to be used.
\end{itemize}

It analyses the problem, identifies risks, discusses possible solutions and justifies the chosen approach.

In summary, the project contract describes what your project will achieve, the technical plan discusses how this will be done. You should be aiming to write 1200-1500 words.

\subsection*{Purpose}
The plan provides:
\begin{itemize}
    \item strategic information for you and your supervisor to be confident that you know what you are doing and a check on the challenge presented by the project
    \item a record of decisions for you to monitor during the project and to review in your evaluation
\end{itemize}

\subsection*{Reviewing your Technical Plan}

\begin{center}
    \begin{tabularx}{\textwidth}{|X|X|}
    
    \hline
    \textbf{Question} & \textbf{Implications of the answer} \\
    \hline
    What are the main artefacts? & Is it clear what you are actually producing?\\
    \hline
    What are the key technical uncertainties? & Areas to be prototyped, topics for investigation\\
    \hline
    What are the characteristics of the system? & DB: data modelling, integrity, performance\linebreak%
    Web: testing on server, UI, maintenance\linebreak%
    Programming: key algorithms, libraries\linebreak%
    Real Time: concurrency issues\\
    \hline
    Can you justify your selections and rejections? & Explain why options have been chosen.\\
    \hline
    Are choices based on familiarity & Choose the best option, not the one that looks easiest.\\
    \hline
    Are clients or special resources involved? & What is plan B when you're let down?
How do you balance their needs against yours?\\
    \hline
    Do you have incremental delivery & If not, what happens if you fall behind schedule?\\
    \hline
    Could someone else do the project from this plan? & If not, consider adding more detail\\
    \hline
    Are there gaps in your knowledge or capabilities? & Explain how you will fill them\\
    \hline
    Are your techniques and approaches off-the-shelf? & Is this project sufficiently challenging?\\
    \hline
    What could kill your project? & Does your plan investigate this early to allow repair\\
    \hline
    \end{tabularx}
\end{center}
\end{document}
